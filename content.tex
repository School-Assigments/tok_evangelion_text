В фильме "Эквилибриум" повествуется о вымышленном государстве Либрия,
правительство которого во главе с Вождем считают, что многие социальные
проблемы (в числе которых и войны) возникают из-за человеческий эмоций,
а значит их надо подавлять, что достигается засчет уничтожения объектов
искусства, принятия всеми гражданами специального препарата и казни
"эмоциональных преступников" --- людей, испытывающих чувства. % чувства -> эмоции?

Главный герой Джон Престон занимает высокий пост в организаци, поддерживающей
установленный порядок. По ходу фильма он проходит путь от человека, полностью
преданного режиму, до важного участника сопротивления и убийцы Вождя. % важного?

С самого начала фильм отсылает зрителя к различным книгам: % произведениям литературы?
сжигание произведений искусства и работа главного героя в организации, этим
занимающейся, напоминает "451 градус по Фаренгейту"\ct{bradbury}; вездесущая
пропаганда, эмоциональные (мысленные) преступления, образ и изображения Вождя
похожи на "1984"\ct{orwell}; само же государство можно считать вариантом
Левиафана Гоббса\ct{hobbes}.

Можно заметить, что к моменту действий картины режим в государстве уже
закрепился, однако же правительство всё равно ведет крайне активную пропаганду,
запрещает всё, что может привести к появлению чувств у граждан, и предписывает
принимать подавляющий эмоции препарат ежедневно. Почему же так происходит?
Стоит предположить, что человечек даже подсознательно не принимает такого рода
ограничения, пытаясь обрести большую свободу. Таким образом, состояние жителей
Либрии сравнимо с понятием неустойчивого равновесия в физике: любое
отклонение от изначального положения приводит к всё большим и большим отклонениям.
