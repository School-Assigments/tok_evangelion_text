В фильме "Эквилибриум" (англ. Equilibrium от лат. aequus "равный" + libra
"весы: равновесие")\footnote{Википедия. Эквилибриум.
(https://ru.wikipedia.org/wiki/Эквилибриум)} повествуется о вымышленном
государстве Либрия, правительство которого во главе с Вождем считают, что
многие социальные проблемы (в том числе и войны) происходят из-за человеческих
эмоций, вследствие чего пытаются их устранить, чтобы общество достигло
состояния равновесия.

Главный герой Джон Престон является так называемым клириком, то есть работает в
организации, занимающейся уничтожением различных произведений искусства и
поимкой "эмоциональных преступников" --- людей, которые испытывают чувства ---
для последующей казни. Такие меры применяются государством для поддержания
действующего порядка.

В самом начале фильма Джон представляется всецело преданным режиму человеком,
исправно выполняющим работу для его сохранения --- он находился в состояния
равновесия. Далее выясняется, что Эррол Партридж, напарник Престона, является
эмоциональным преступником, и клирику приходится застрелить того на месте.
Перед смертью Эррол прочитал стихотворение, которое как бы "выбило" Престона из
равновесия: он начинал задумываться о необходимости подчинятся режиму,
перестал безупречно выполнять свою работу.

Таким образом, выражаясь физическими терминами, равновесие, в котором пребывал
Джон было неустойчивым: после небольшого отклонения от изначального положения,
клирик стал отклоняться всё больше и больше, что в конечном счете привело к
объединению с повстанцами и успешному покушению на Вождя.

Казалось бы, в таком неустойчивом состоянии пребывал лишь Джон (или хотя бы
небольшая часть населения), однако же это, вероятно, ошибочное суждение, а
в названном состоянии пребывает абсолютное большинство.
К такой мысли можно прийти, если рассмотреть политику Либрии по отношению
к гражданам: государство пресекает даже малейшие проявления чувств или
культуры, заставляет жителей ежедневно принимать подавляющий эмоции препарат.
То есть правительство знает, что сдвинуть народ с "мертвой точки" очень легко и
что личность будет стремиться к свободе и независимости от государственной машины.
К тому же, по словам одного из повстанцев, для начала революции будет достаточно
отключить пропаганду и поступление препарата лишь на один день.

Во многом авторская позиция Эквилибриума схожа с идеями, содержащимися в книгах
"451 градус по Фаренгейту"\ct{bradbury} и "1984"\ct{orwell} (стоит отметить
наличие явных отсылок на названные произведения и в самом фильме), однако же она
сильно отличается и даже противостоит мыслям Томаса Гоббса\ct{hobbes}.
Действительно, Гобсс, как и правительство Либрии, видит решение человеческих
проблем в Левиафане --- сильном государстве --- в то время как Уиммер эту идею
оспаривает.

С позицией автора я согласен, и фильм в основном мне понравился, однако же
некоторые моменты кажутся странными, как то: игра актеров, изображающих
безэмоциональных персонажей, временами оказывается слишком эмоциональной, что
несколько портит впечатление о картине.
